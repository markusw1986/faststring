\documentclass[12pt]{report}


\usepackage[T1]{fontenc}
\usepackage[utf8]{inputenc} 
\usepackage[ngerman]{babel}

\usepackage{setspace}
\usepackage{xcolor}
\usepackage{listings}
\usepackage{algorithm}
\usepackage{algorithmic}
\usepackage{float}
\usepackage{amsfonts}
\usepackage{graphicx}
\usepackage{titlesec}
\usepackage[linktoc=all]{hyperref}
\hypersetup{
    colorlinks,
    citecolor=black,
    filecolor=black,
    linkcolor=black,
    urlcolor=black
}


\lstset{basicstyle=\ttfamily,breaklines=true}

\definecolor{javared}{rgb}{0.6,0,0} % for strings
\definecolor{javagreen}{rgb}{0.25,0.5,0.35} % comments
\definecolor{javapurple}{rgb}{0.5,0,0.35} % keywords
\definecolor{javadocblue}{rgb}{0.25,0.35,0.75} % javadoc
 
\lstset{language=Java,
basicstyle=\footnotesize\ttfamily,
keywordstyle=\color{javapurple}\bfseries,
stringstyle=\color{javared},
commentstyle=\color{javagreen},
morecomment=[s][\color{javadocblue}]{/**}{*/},
%numbers=left,
%numberstyle=\tiny\color{black},
%stepnumber=1,
%numbersep=10pt,
tabsize=4,
showspaces=false,
showstringspaces=false}

\lstset{literate=%
    {Ö}{{\"O}}1
    {Ä}{{\"A}}1
    {Ü}{{\"U}}1
    {ß}{{\ss}}1
    {ü}{{\"u}}1
    {ä}{{\"a}}1
    {ö}{{\"o}}1
    {~}{{\textasciitilde}}1
}


\lstdefinelanguage{Bytecode}{
  keywords={GETSTATIC, INVOKEVIRTUAL, LDC, RETURN, ICONST, ILOAD, IADD, ISTORE},
  keywordstyle=\color{black}\bfseries,
  identifierstyle=\color{black},
  sensitive=false,
  comment=[l]{//},
  morecomment=[s]{/*}{*/},
  commentstyle=\color{purple}\ttfamily,
  stringstyle=\color{red}\ttfamily,
  morestring=[b]',
  morestring=[b]",
  basicstyle=\footnotesize\ttfamily
}

\setcounter{secnumdepth}{2}

\titleformat{\paragraph}{\normalfont\normalsize\bfseries}{\theparagraph}{1em}{}
\titlespacing*{\paragraph}{0pt}{3.25ex plus 1ex minus .2ex}{1.5ex plus .2ex}

\title{Optimierung mittels der Auswahl von String Repräsentationen in Java Bytecode}
\author{Markus Wondrak\\Goethe Universität Frankfurt am Main\\http://www.sepl.informatik.uni-frankfurt.de}
\date{\today}

\begin{document}
\maketitle
\pagenumbering{roman}
\tableofcontents
\listoffigures
\listofalgorithms

\begin{abstract}
Es geht um blibla blubb
\end{abstract}
\pagenumbering{arabic}


\chapter{Einleitung}
\pagenumbering{arabic}
Die Verwendung von optimierten Datentypen beim Entwickeln eines Programms ist schwierig.
Die eingesetzten Implementierungen von Standarddatentypen decken zumeist allgemeine 
Fälle ab, sind aber nicht optimal für spezielle Szenarien. Zudem sind spezielle
optimierte Datentypen den Anwendern meist nicht oder zumindest nur begrenzt bekannt.
So wird bei der Entwicklung von Software einer aus den Datentypen 
gewählt, die dem Entwickler bekannt sind. Dabei ist nicht nur die fehlende 
Kenntnis über diese optimierten Typen ein Grund für das Übergehen eben dieser. Die 
Entscheidungsfindung bis hin zur Wahl für oder gegen einen speziellen
Typ verzögert die Arbeit während der Entwicklung von Systemen. 

Ein weiteres Problem ergibt sich bei der Betrachtung von Altsystemen. Bei bereits 
lang bestehenden Anwendung, besteht das Problem der Wartbarkeit von eingesetzten Datentypen. 
Durch die Weiterentwicklung von eingesetzten Bibliotheken können Optimierungspotentiale
erst nach der aktiven Entwicklung von einzelnen Modulen entstehen. Hierbei ergibt sich
wieder das Problem, dass diese Potentiale bekannt und auch richtig umgesetzt werden müssten.

Ziel dieser Arbeit ist es daher zu untersuchen, ob sich Programme durch Auswahl und 
Substitution von alternativen Datentypen automatisiert optimieren lassen. Zu diesem Zweck
soll ein System erstellt werden, das durch statische Code Analyse und automatische
Transformation eine Optimierung eines Programms durchführt \footnote{Der Quellcode zu 
dem System (inklusive Tests) sowie die Benchmarks und Auswertungsskripte sind unter 
\texttt{http://github.com/wondee/faststring} zu finden.}.

Diese Untersuchung kann allerdings keine allumfassende Auswertung bereitstellen und wird 
sich auf die Optimierung der Laufzeit des \texttt{java.lang.String} Datentyps der 
\textit{Java} Plattform und der Programmiersprache \textit{Java} beschränken. Nach 
der Auswertung soll eine Aussage darüber möglich sein, ob diese Optimierungen möglich sind.
Daher sollen die beiden Hypothesen untersucht werden:
\\
\begin{enumerate}
	\item Der String Datentyp bietet viele Möglichkeiten der Optimierung.
	\item Das automatische Ersetzen von Standard-Datentypen durch alternative Datentypen führt zu einem
	Performance Gewinn.
\end{enumerate}

Im Rahmen dieser Arbeit soll ein System erstellt werden, das String Operationen in einem
gegebenen Java Programm mit einer entsprechenden optimierten Version auf Basis des 
Java Bytecodes ersetzt. Dabei soll das System anhand statischer Code Analyse jene Stellen
lokalisieren, an denen eine Optimierung angewendet werden kann, und mittels der Transformation des 
Bytecodes des Programms diese Optimierungen anwenden. Als Ergebnis wird ein lauffähiges
Programm erwartet, dass bei gleicher Eingabe eine geringere Ausführungszeit besitzt 
als das originale Programm.

Es existieren Arbeiten, die Optimierungen ebenfalls durch die Auswahl von alternativen
Datentypen anstreben \cite{coco, cham, brain}. Diese setzen entgegen der in dieser Arbeit 
verwendeten statischen Code Analyse auf eine dynamische und beschränken sich auf die Auswahl von 
Container Typen, die zur Laufzeit ersetzt werden, um eine Optimierung des Programms zu erreichen.

In Kapitel 2 werden die Werkzeuge und Grundlagen beschrieben, auf denen das in dieser Arbeit
entwickelte System aufbaut. Darauf folgt Kapitel 3, welches die optimierten String Typen vorstellt
die in dieser Arbeit entwickelt wurden. Kapitel 4 erläutert den Analyse Prozess des Systems, indem
die Datenstrukturen sowie der Algorithmus für die statische Code Analyse beschrieben wird.
In Kapitel 5 wird basierend auf den Analyse Ergebnissen die Transformation des Bytecodes 
präsentiert. Kapitel 6 widmet sich der Auswertung der Tests in Form von Benchmarks
und begründet die Ergebnisse. Kapitel 7 zieht ein Fazit und beschreibt mögliche
Weiterentwicklungen des Systems.    
\chapter{Werkzeuge}

In den folgenden Abschnitten sollen die verwendeten Werkzeuge kurz vorgestellt werden. Dabei handelt es 
sich zum einen um den Java Bytecode, als auch um die Software Bibliothek \textit{WALA}, auf deren API das 
von mir entwickelte System basiert. 

\section{Java Bytecode}

Die Plattformunabhängigkeit, die in Java geschriebenen Programmen zugesprochen wird, ist vorallem mit der 
Rolle der Java  Virtual Machine zu erklären. Java Programme werden in einen Zwischencode, den Java Bytecode, 
übersetzt, welcher von der System spezifischen JVM ausgeführt wird. Dabei ist Programmiersprache Java nicht 
die einzige in Bytecode übersetzbare Sprache. Es existieren neben den bekanntesten Scala, Jython, Groovy,
JavaScript noch viele weitere. Einmal in Bytecode übersetzt in diesen Sprachen geschriebene Programme auf 
jeder der Java Spezifikation entsprechenden JVM auführen. 

Bytecode ist eine Sammlung von Instruktionen welche durch \textit{opcodes} von 2 Byte Länge definiert 
werden. Zusätzlich können noch 1 bis $n$ Parameter verwendet werden. Die Sprache ist Stack-orientiert, das 
bedeutet, dass von Operationen verwendete Parameter über einen internen Stack übergeben werden. Als Beispiel 
dient der folgende Bytecode:

\begin{lstlisting}[language=Bytecode]
	ICONST 5 	// legt den konstanten int Wert 5 auf den Stack 
	ILOAD 1		// läd die lokale integer Variable 1 und legt sie auf den Stack
	IADD 		// addiert die ersten beiden Werte auf dem Stack und legt das Ergebnis auf den Stack
	ISTORE 2	// speichert den Wert auf dem Stack in der Variable 2
\end{lstlisting} 

Dabei existiert der Stack nur als Abstraktion für den eigentlichen Prozessor im Zielsystem. Wie die 
jeweilige JVM den Stack in der Ziel Plattform umsetzt ist nicht definiert. Die Instruktionen lassen sich in 
folgende Kategorien einordnen:

\begin{itemize}
	\item Laden und Speichern von lokalen Variablen (\texttt{ILOAD}, \texttt{ISTORE})
	\item Arithmetische und logische ausdrücke (\texttt{IADD})
	\item Object Erzeugung bzw. Manipulation (\texttt{NEW}, \texttt{PUTFIELD})
	\item Stack Verwaltung (\texttt{POP}, \texttt{PUSH}) 
	\item Kontrollstruktur (\texttt{IFEQ}, \texttt{GOTO})
	\item Methoden Aufrufe (\texttt{INVOKEVIRTUAL}, \texttt{INVOKESTATIC})
\end{itemize}  


\section{WALA}

Bei \textit{WALA} handelt es sich um die "T.J. Watson Library for Analysis". Eine von ehemals von IBM 
entwickelte Bibliothek für die statische Codeanalyse von Java- und JavaScript Programmen. Das Framework 
übernimmt dabei das Einlesen von \textit{class} Dateien und stellt eine Repräsentation, die sogenannte 
\textit{Intermediate Representation}, des Bytecodes zur Verfügung. Diese IR stellt die zentrale 
Datenstruktur dar und soll in diesem Abschnitt detailiert beschrieben werden.

Für die Manipulation des Bytecodes existiert innerhalb des Frameworks ein Unterprojekt, das diese Aufgabe 
übernimmt: Shrike. Im Zweiten Abschnitt soll diese API kurz vorgestellt werden.  

\subsection{IR}

Die \textit{Intermediate Representation} (IR) eine Abstraktion zum Stack basierten Bytecode. Ein IR ist
in Single Static Assignment Form, welche sich dadurch auszeichnet, dass jeder Variablen immer genau 
\textbf{einmal} ein Wert zugewiesen wird. Zusätzlich besteht die IR aus dem Kontroll Fluß Graphen der 
Methode, welcher wiederrum aus Basic Blocks zusammengesetzt ist. Ein Basic Block ist eine Zusammenfassung 
von aufeinander Folgende Instruktionen, welche in jedem Fall nach einander ausgeführt werden.

Die Variablen innerhalb des IRs nennt WALA \textit{value numbers}. Diese beziehen sich immer auf eine 
Referenz, allerdings kann sich eine Referenz sich auf mehrere tatsächliche value numbers in der IR beziehen.
Dies folgt aus der SSA Form, wird eine Variable im Bytecode zweimal ein Wert zugewiesen, wird diese 
doppelte Zuweisung in der SSA Form durch das Einführen einer neuen value number entfernt. Die Operationen
werden auch nur mit Bezug auf die value numbers beschrieben.   
 
Da die Zwischendarstellung vom Stack abstrahieren soll, werden auch alle Operationen, die den Stack betreffen
(wie z.B. \texttt{LOAD}, \texttt{STORE}, \texttt{PUSH} oder \texttt{POP}) nicht nicht mit in diese 
Repräsentation übernommen. Dabei werden die Bytecode Indices der überig gebliebenen Instruktionen 
berücksichtigt und alle anderen Stellen in dem beinhaltenden Array mit \texttt{null} Werten aufgefüllt.
Instruktionen werden von Objekten vom Typ \texttt{SSAInstruction} und dessen Untertypen dargestellt.

Die Verwaltung der value numbers wird von einem Typ namens \texttt{SymbolTable} übernommen. Es kommt bei der
IR Erstellung zum Einsatz, wenn bei der Simulation des Bytecodes neue Variablen verwendet werden, um neue 
value numbers zu erzeugen.

Aufgrund der SSA Form der IR lässt sich für jede value number genau eine Definition bestimmen. Zu diesem 
Zweck bietet WALA den Typ \texttt{DefUse} an, welcher für jedes IR-Objekt erstellt werden kann. Er ermöglicht
einen einfachen Zugriff aus die Instruktionen, die eine value number definiert (\textit{def}; z.B. als 
Rückgabe aus einem Methodenaufruf), und eine Menge an Instruktionen, die die entsprechende value number 
verwenden (\textit{use}; z.B. als Rückgabewert in einem \texttt{RETURN} Statement).
   
Das IR und das dazugehörige DefUse Objekt werden in dem System internen Datentyp \texttt{AnalyzedMethod} 
zusammengefasst.

% TODO phis

\subsubsection{Anpassungen}

In WALA werden beim Erstellen des IR für alle Konstanten mit demselben Wert dieselbe value numbers erzeugt. 
Da für die \textit{Analyse} verschiedenen Referenzen getrennt getrennt untersucht werden mussten, wurde 
für die Erzeugung einer value number für eine Konstante der eingebaute caching Mechanismus umgangen. 

Darüber hinaus war für die \textit{Transformation} die Information nötig, an welcher Stelle im Bytecode eine
entsprechende Konstante erzeugt wird (z.B. mittels \texttt{LCD}). Um dies zu Erreichen wurde dieser Bytecode 
Index während dem Durchlaufen der Instrukionen innerhalb der \texttt{SymbolTable} gespeichert, sodass er 
beim Klienten des IRs zur Verfügung steht.

Da diese Änderungen nicht in den Haupt Branch von WALA eingepflegt werden durften, benötigt das System den 
Fork des WALA Projektes \footnote{Dieser ist unter \texttt{http://github.com/wondee/WALA} zu finden.}.


\subsection{Shrike}

Shrike ist ein Unterprojekt innerhalb des WALA Frameworks. Shrike übernimmt dabei das Lesen und das 
Schreiben von Bytecode aus bzw. in class Dateien. Dabei wird es zum einen beim Erstellen eines IR aus einer 
Methode verwendet, zum Anderen bietet es eine "Patch-based" API an um den Bytecode einer eingelesenen 
Methode zu verändern. Dies geschieht über das Einfügen von \texttt{Patch}es, welche über einen 
entsprechenden \texttt{MethodEditor} überall im Bytecode einer Methode eingefügt werden oder auch 
ursprüngliche Instruktionen komplett ersetzen. Zusätzlich enthält es einen \texttt{Verifier}, der erzeugten 
Bytecode überprüft, so dass ungültige Stack Zustände oder Typfehler noch während der Manipulation erkannt 
werden können. 

In dem von mir entwickelten System werden alle Bytecode Manipulationen mit Hilfe von Shrike umgesetzt. 


\chapter{Analyse}

Das Folgende Kapitel beschreibt den Analyse Algorithmus, des von mir entworfenen Systems.
Im ersten Abschnitt sollen die verwendeten Datenstrukturen vorgestellt und beschrieben werden.
Der zweite Abschnitt beschreibt den eigentlichen Algorithmus.

\section{Datenstrukturen}

Für den Algorithmus wurden zwei grundlegende Datenstrukturen entworfen. Der 
\textit{Datenflussgraph} repräsentiert den Datenfluss der Referenzen innerhalb 
einer Methode und wird im ersten Abschnitt vorgestellt. Zu optimierende Referenzen
werden in diesem Graphen mit sogenannten \textit{Labels} versehen. Dieser Datentyp
soll im zweiten Abschnitt beschrieben werden.

\subsection{Datenfluss Graph}\label{ssec:DFG}

Eine auf einem IR basierende Methode wird vom System mittels eines Datenflussgraphen 
repräsentiert. Dieser wird vor der eigentlichen Analyse aus einer gegeben
Methode und einer Menge an initialen Referenzen vom \texttt{DataFlowGraphBuilder}
erzeugt. Der Graph ist gerichtet und setzt sich aus zwei verschiedenen Knoten zusammen:

\begin{itemize}
	\item \texttt{Reference}: eine value number aus dem IR
	\item \texttt{InstructionNode}: eine Instruktion aus dem IR
\end{itemize}

Sei im Folgenden der Datenflussgraph $G = (V, E)$, $R$ die Menge aller \texttt{Reference} 
Knoten und $I$ die Menge aller \texttt{InstructionNode}s. 
\\
Im Graph gilt $\forall (x, y) \in V,  (x \in R \wedge y \in I) \vee (x \in I \wedge y \in X)$.
Eine Kante $i \in I, r \in R, (i, r)$ beschreibt eine \textit{Definition}, die aussagt, 
dass die Referenz $r$ durch die Instruktion $i$ definiert wird. Ein Kante 
$i \in I, r \in R, (r, i)$ ist eine \textit{Benutzung} (im folgenden \textit{Use}
genannt). 

\texttt{Reference} Knoten werden für jede betroffene value number erzeugt. Für
die Erstellung von \texttt{InstructionNode} Objekten steht die 
\texttt{InstructionNodeFactory} zur Verfügung, die für eine gegebene 
\texttt{SSAInstruction} eine entsprechende \texttt{InstructionNode} erstellt. Um
für dieselbe \texttt{SSAInstruction} immer dasselbe \texttt{InstructionNode} 
Objekt zu garantieren verwendet die Factory einen internen Cache, der eine 
Abbildung $SSAInstruction \rightarrow InstructionNode$ verwaltet und für jede 
\texttt{SSAInstruction} prüft ob für diese bereits eine \texttt{InstructionNode} 
erstellt wurde.

Die Erstellung eines \texttt{DataFlowGraph}s beginnt immer mit einer Menge an 
initialen \texttt{Reference} Objekten. Ausgehend von dieser Startmenge werden über 
das \texttt{DefUse}-Objekt des betroffenen IRs die Definition und alle Uses in den 
Datenflussgraphen eingefügt. Algorithmus \ref{alg:dfg} beschreibt die Erstellung des 
Graphen. Die verwendete Queue Implementierung ist eine auf einem 
\texttt{java.util.LinkedHashSet} basierende Eigenentwicklung mit dem Namen
\texttt{de.unifrankfurt.faststring.analysis.util.UniqueQueue}.

\begin{algorithm}[H]
	\caption{Erstellung des Datenflussgraphen}\label{alg:dfg}
	\begin{algorithmic}[1]
		\STATE $q \gets$ \texttt{new Queue($initialReferences$)}
		\STATE $g \gets$ \texttt{new DataFlowGraph()}
		\WHILE {\texttt{not $q$.isEmpty()}}
			\STATE $r \gets q$\texttt{.remove()}
			\IF {\texttt{not $g$.contains()}}

				\STATE $def \gets$ \texttt{$defUse$.getDef($r$)}
				\STATE $uses \gets$ \texttt{$defUse$.getUses($r$)}

				\STATE \texttt{$newInd$.add($def$)}
				\STATE \texttt{$newInd$.add($uses$)}

				\STATE \texttt{$r$.setDef(factory.create($def$))}

				\FOR{\texttt{$ins \in defUse$.getUses($r$)}}
					\STATE \texttt{$r$.addUse(factory.create($ins$))}
				\ENDFOR
				\FOR{$ins \in newIns$}
					\STATE \texttt{$q$.add($ins$.getConnectedRefs())}
				\ENDFOR
				\STATE \texttt{$g$.add($r$))}
			\ENDIF
		\ENDWHILE
		\RETURN $g$
	\end{algorithmic}
\end{algorithm}

Jede \texttt{InstructionNode} besitzt eine Definition, die Nummer der Referenz 
die diese Instruktion erzeugt und eine Liste von Uses, die Nummern der Referenzen 
die es benutzt. Darüber hinaus noch Informationen zu Bytecode Spezifika, die 
im Kapitel \ref{sec:locals} betrachtet werden.

Für verschiedene \texttt{SSAInstruction} Typen existieren entsprechende 
\texttt{InstructionNode} Subtypen. Allerdings gibt es auch Typen die nicht
einer \texttt{SSAInstruction} zugeordnet werden können. Im Folgenden sollen die
wichtigsten Knotentypen vorgestellt werden. Es existieren darüber hinaus weitere
für die das System zur Zeit keine Unterstützung bietet, da es ausschließlich für 
String Typen und komplexe Objekte ausgelegt ist.

\subsubsection{MethodCallNode}

Eine \texttt{MethodCallNode} repräsentiert einen Methoden Aufruf. Es besitzt, wenn
vorhanden, eine Definition, welche den Rückgabewert repräsentiert, einen Receiver, 
wenn es keine statische Methode ist und eine Liste an Parametern. Zusätzlich die 
aufgerufene Methode. 

\subsubsection{ContantNode}

Dieser Knoten Typ stellt eine Konstanten Definition dar. Er besitzt ausschließlich 
die Definition, welcher Referenz diese Konstante zugewiesen wird. Für diesen Typ
existiert keine entsprechende \texttt{SSAInstruction}.

\subsubsection{ParameterNode}

Die \texttt{ParameterNode} stellt eine Definition eines Parameters der Methode dar.
Wird eine Variable innerhalb der Methode als Parameter in der Methoden Signatur
deklariert, wird deren Definition als \texttt{ParameterNode} im Datenflussgraphen
repräsentiert. Für diesen Typ existiert keine entsprechende \texttt{SSAInstruction}.

\subsubsection{NewNode}

Dieser Typ entspricht einer \texttt{NEW} Anweisung, die ein neues Objekt eines 
gegebenen Typen erstellt. Es besitzt eine Definition und den Typ des instanziierten
Objekts. 

\subsubsection{ReturnNode}

\texttt{ReturnNode} Typen sind \texttt{RETURN} Anweisungen. Die besitzen 
ausschließlich eine Referenz als Use. Diejenige Referenz, die sie aus der Methode 
zurückgeben. Dieser Typ kann keine Definition darstellen.

\subsubsection{PhiNode}

Die \texttt{PhiNode} steht für eine $\phi$-Instruktion aus dem IR. Sie besitzt eine
Referenz als Definition und 2 bis $n$ Uses.

\subsection{Label}

Das \textit{Label} entspricht einer Markierung, mit der Knoten in einem 
Datenflussgraphen versehen werden können. Dabei steht ein Label (oder 
\texttt{TypeLabel}, wie der Datentyp im System heißt) für einen Optimierten Typ.
Die Semantik hinter einem markierten Knoten ist, dass diese Referenz bzw. Instruktion
durch den entsprechenden Optimierten Typ ersetzt werden kann.

Es kann nicht für alle \texttt{InstructionNode}s ein Label gesetzt werden. Genauer
gesagt lassen sich ausschließlich für die Knotentypen \texttt{MethodCallNode}, 
\texttt{NewNode} und \texttt{PhiNode} ein Label setzen, da sich nur diese 
Instruktionen in einen optimierten Typ umwandeln lassen.

Das \texttt{TypeLabel} beinhaltet alle Regeln, die für die Verwendung eines
Optimierten Typen existieren. Dazu gehören

\begin{itemize}
	\item der Originale, sowie der Optimierte Typ
	\item die Methoden für die Optimierungen im optimierten Typ angeboten werden
	\item alle Methoden die darüber hinaus vom Optimierten Typ unterstützt werden
	\item Methoden, die den optimierten Typ als Rückgabewert zurückgeben
	\item kompatible Label
\end{itemize}

Dabei ist diese Liste bereits eine Abstraktion zu den Methoden, die das Interface 
besitzt. Im System lassen sich Label Definition auf 2 Arten erstellen:

\begin{enumerate} 
	\item Durch das Implementieren des Interfaces \texttt{TypeLabel}
	\item Durch das Erstellen einer \texttt{.type} Datei
\end{enumerate}

Zwar unterstützt das Kommandozeilen Tool zur Zeit nur die zweite Variante,
programmatisch lässt sich allerdings auch die erste Alternative umsetzen. Im 
Folgenden sollen die beiden Möglichkeiten zur Definition eines \texttt{TypeLabel}s
betrachtet werden.

\subsubsection{Das TypeLabel Interface} \label{sssec:typeLabel}

Das Interface beinhaltet alle Methoden, die der Analyse- und Transformationsprozess
benötigt. In diesem Kapitel sollen zunächst nur die Methode betrachtet werden, die
für den Analyse Algorithmus verwendet werden, die Übrigen werden im Abschnitt 
\ref{ssec:infoLabel} betrachtet. 

\begin{description}
	\item[\texttt{canBeUsedAsReceiverFor(MethodReference)}] Legt fest, ob eine 
	markierte Referenz als Empfänger für den übergebenen Methodenaufruf dienen kann.
	\item[\texttt{canBeUsedAsParamFor(MethodReference,int)}] Bestimmt, ob eine 
	markierte Referenz als Parameter in dem gegebenen Methodenaufruf an der 
	entsprechenden Stelle (der \texttt{int} Parameter) verwendet werden kann.
	\item[\texttt{canBeDefinedAsResultOf(MethodReference)}] Sagt aus, ob die 
	gegebene Methode einen optimierten Typ zurückgeben kann. Dies impliziert, dass
	der Methodenaufruf selber auch markiert ist.
	\item[\texttt{findTypeUses(AnalyzedMethod)}] Gibt eine Menge an \texttt{Reference}
	Objekten zurück, auf denen innerhalb der gegebenen Methode eine der von 
	der Optimierung betroffenen Methode aufgerufen wird. Für diesen Algorithmus 
	existiert bereits eine Implementierung in der Klasse \texttt{BaseTypeLabel}.
	\item[\texttt{compatibleWith(TypeLabel)}] Gibt an, ob das übergebene Label 
	kompatibel mit diesem Label ist.
\end{description}

Alle diese Methoden werden von den \texttt{InstructionNode} Implementierungen 
verwendet. Wie genau das passiert wird im Abschnitt \textit{Algorithmus} beschrieben.

\subsubsection{Das .type Dateiformat}

Da das Implementieren des Interfaces eher komplex ist, wurde für die einfachere 
Definition eines Types ein Datei Format entwickelt, welches von der Komplexität des
Interfaces abstrahieren soll. In dieser werden nicht die Regeln selbst, sondern 
die Fakten beschrieben, aus denen die Regeln für den Algorithmus hergeleitet werden 
können, beschrieben.

Aus einer Datei im \texttt{type} Format wird mittels eines internen Parsers ein
\texttt{TypeLabelConfig} Objekt erzeugt, welches als \texttt{TypeLabel} Objekt für
den Algorithmus fungiert.

Für die inhaltliche Struktur der Datei wurde JSON (JavaScript Object Notation) 
gewählt eine Darstellung anzubieten, die sowohl für Menschen als auch für das 
Programm leicht zu lesen und zu verstehen ist. Die Attribute innerhalb der Datei 
werden im Folgenden beschrieben:

\begin{description}
	\item[name] Der Name des Labels
	\item[originalType] Der voll qualifizierte Name des zu ersetzenden Typs
	\item[optimizedType] Der voll qualifizierte Name des zu optimierten Typs
	\item[methodDefs] Liste von Methoden, diesen wird eine ID vergeben um sie im 
	folgenden über diese ID zu referenzieren. Ein Eintrag in dieser Liste setzt
	sich zusammen aus:
	\begin{description}
		\item[id] eine eindeutige ID für die diese Methode
		\item[desc] die Beschreibung dieser Methode. Dies ist ein eigenes Objekt 
		und besteht aus den Attributen:
		\begin{description}
			\item[name] der Name der Methode
			\item[signature] der Signatur der Methode. Zusammengesetzt aus der
			Parameterliste und der Rückgabewert. Die Typen müssen dabei in der 
			internen JVM Form angegeben werden. (Beispiel: "\texttt{(I)Ljava/lang/String;}"
			, ein Parameter vom Typ \texttt{int} und Rückgabewert vom Typ
			\texttt{java.lang.String})
		\end{description}	
	\end{description}
	\item[effectedMethods] Liste von Methoden IDs. Für diese Methoden existieren 
	optimierte Varianten in dem optimierten Typen.
	\item[supportedMethods] Liste von Methoden IDs. Diese Methoden werden auch vom
	optimierten Typ unterstützt. Es handelt sich bei diesen aber nicht um Optimierungen.
	\item[producingMethods] Liste von Methoden IDs. Alle diesen Methoden erzeugen 
	in ihrer optimierten Variante optimierte Typen.
	\item[compatibleLabel] Liste von Strings. Alle Labels die mit diesem Label
	kompatibel sind.
	\item[parameterUsage] Ein Objekt. Dabei ist jeder key die ID einer Methode 
	und der entsprechende value eine Liste von Ganzzahlen. Ein Eintrag bedeutet, 
	dass diese Methode mit einem Optimierten Typ als Parameter mit diesem Index 
	umgehen kann. 
	\item[optimizedParams] Ein Objekt. Dabei ist jeder key die ID einer Methode 
	und der entsprechende value eine Parameter Liste in interner JVM Notation. 
	Wird bei einer Optimierung einer Methode eine andere Parameter Liste erwartet
	als die ursprüngliche der originalen Methode, so muss die neue Signatur an dieser
	Stelle angegeben werden.
	\item[staticFactory] Ein String. Der Name einer statischen Factory Methode mit 
	einem Übergabeparameter vom Typ des Originalen Typs. Diese muss einen entsprechenden
	Optimierten Typ zurückgeben.
	\item[toOriginalType] Ein String, Der Name einer Methode ohne Parameter, die
	aus dem optimierten Objekt, ein entsprechendes vom Originalen Typ zurückgibt.

\end{description}

\subsection{Konventionen}\label{subs:conventions}

Bei dem Erstellen von optimierten Typen sind einige Konventionen zu befolgen. Das 
System rechnet damit, dass diese Annahmen befolgt werden. Diese Regeln sind im Folgenden
beschrieben.

\subsubsection{Methodennamen}

Wird eine Methode als \textit{effectedMethod} deklariert, wird der optimierte Gegensatz
durch den Namen identifiziert. Wenn z.B. die Methode \texttt{f()} aus dem Typ \texttt{A}
optimiert werden soll, so muss in dem optimierten Typ \texttt{AOpt} eine Methode
\texttt{f()} existieren. Besitzt die originale Methode eine Parameterliste, so muss
diese mit derer der optimierten Methode in Anzahl und Typen übereinstimmen. Eine Ausnahme
bildet dabei die Verwendung von ebenfalls optimierten Parametern. In diesem Fall muss im Feld 
\textit{optimizedParams} in der \textit{type} Datei ein entsprechender Eintrag erfolgen.


\section{Algorithmus}

In diesem Abschnitt soll die Idee hinter dem Analyse Algorithmus sowie dessen
Implementierung vorgestellt werden. Hierzu wird zunächst das Konzept der "Bubble"
erläutert um danach die Umsetzung dieses Konzepts im eigentlich Algorithmus zu 
betrachten.

\subsection{Motivation}

Optimierte Referenzen sind im Falle von String Objekten nicht kompatibel mit den 
Originalen. So treten Probleme in den Folgenden Szenarien auf:

\begin{description}
	\item[Referenz als Methoden Parameter] Die Signatur der 
	optimierten Methode, wird nicht verändert, da es dazu führen würde, dass
	Klienten Code, der diese Methode weiterhin mit dem originalen Typ aufruft, 
	nicht mehr kompilieren würde.
	\item[Referenz als Rückgabewert] Ein ähnliches Problem existiert, wenn die 
	Referenz zurückgegeben wird. Die Signatur der Methode definiert den Originalen
	Typ als Rückgabetyp und das zurückgeben eines anderen Typs als eben dieser 
	definierte würde zu Laufzeitfehlern führen.
	\item[Methodenaufruf auf Referenz] Wird diese optimierte Referenz als Empfänger 
	von einer Methode verwendet, die nicht zu den optimierten oder unterstützen
	Methoden dieses optimierten Typs gehört, würde es zu Laufzeitfehlern kommen, 
	da diese aufzurufende Methode nicht im optimierten Typ vorhanden ist. 
	\item[Feld Zugriff (\texttt{GETFIELD}, \texttt{PUTFIELD})] Wird die optimierte 
	Referenz in ein oder aus einem Feld innerhalb eines Objektes (oder in ein 
	statisches Feld einer Klasse) gesetzt, stimmt auch in diesem Szenario der Typ
	des optimierten und des originalen Objekts nicht überein.
	\item[Array Zugriff] Bei dem Zugriff auf ein Array, sowohl lesend als auch 
	schreibend, existiert eine Unstimmigkeit mit dem Typ des Arrays.
	\item[Referenz Aufruf Parameter] Wird die Referenz als Parameter für einen
	Methoden Aufruf verwendet und diese Methode ist nicht in der Label Definition
	als Methode deklariert, die mit einem optimierten Typ umgehen kann, dann 
	erwartet diese Methode den originalen Typ und nicht den optimierten.
\end{description}

Erweitert allerdings der optimierte Typ seinen originalen Typ so wäre, durch den 
Polymorphismus, der optimierte Typ genauso wie der originale verwendbar. Allerdings 
ist der Typ \texttt{java.lang.String} final, was bedeutet, dass von diesem Typ nicht 
abgeleitet werden kann. Darüber hinaus bieten sich Ableitungen für Optimierungen nicht
an, da das dynamische dispatchen zusätzlichen Aufwand für die JVM darstellt, da 
zunächst die Implementierung des Methode zu lokalisieren.

Aus diesem Grund müssen in den Code Konvertierungen in und vom optimierten Typ 
eingefügt werden. Eine Konvertierung zum optimierten Typ muss vor dem zu optimierenden
Methodenaufruf erfolgen. Eine Umwandlung vom optimierten Typ allerdings muss vor einer 
nicht kompatiblen Benutzung dieser Referenz erfolgen um bei dieser den originalen Typ 
zu verwenden.

\subsection{Die "Bubble"}

Da Konvertierungen zusätzliche Laufzeit erfordern, muss es das Ziel sein die Anzahl 
der durchgeführten Konvertierungen zu minimieren. Dies wird erreicht indem man den
Bereich, in dem ein optimierter Typ statt des originalen innerhalb der Methode 
verwendet maximiert. Dieser Bereich, in dem ein optimierter, statt des originalen, 
Typs für eine Referenz verwendet wird, wird im Folgenden \textit{Bubble} genannt. 

Die Bubble entsteht mittels der Markierung von Knoten im Datenflussgraphen. Es wird 
für jede gegebene Instanz vom Typ \texttt{TypeLabel} eine Bubble auf dem Graphen 
erzeugt. Dabei kann ein Knoten immer nur mit einem Label markiert sein, daher kann
ein Knoten immer nur Teil einer Bubble sein.

Ziel des Algorithmus ist es diese Bubble so groß wie zu definieren. Als Anfangszustand 
werden alle zu optimierenden Methodenaufrufe mit dem zu verarbeitenden Label markiert.

\subsection{Umsetzung des Algorithmus}\label{ssec:umAlg}

Als Eingabe für den Algorithmus dient ein Objekt vom Typ \texttt{DataFlowGraph}, der 
den bereits beschriebenen Datenflussgraphen darstellt. Zum Erstellen des Graphen werden
wie in \ref{ssec:DFG} beschrieben zunächst initiale Referenzen benötigt. Diese Referenzen
werden über die Methode \texttt{Set<Reference>} \\ \texttt{findTypeUses(AnalyzedMethod)} ermittelt, 
welche sich in der abstrakten Klasse \texttt{BaseTypeLabel} befindet. Diese Methode ist auch Teil des 
\texttt{TypeLabel} Interfaces. Grundlage dieser Ermittlung sind die von der in der Label 
Definition definierten \textit{effectedMethods}, also derjenigen Methoden für die dieses 
Label eine Optimierung darstellt. Die \texttt{findTypeUses} Methode erstellt für jede 
Referenz, auf der in der gegebenen Methode eine der \textit{effectedMethods} aufgerufen 
wird, ein \texttt{Reference} Objekt. Dabei wird für jede value number genau ein Referenz 
Objekt erzeugt. Die erstellten Referenzen werden dem verwendeten Label markiert.

Die Implementierung des Algorithmus befindet sich als statische Methode in der Klasse 
\texttt{LabelAnalyzer}. Ausgehend von den initial markierten Knoten, wird jeweils die 
Definition sowie alle Uses einer Referenz betrachtet, ob diese mit dem Label der Referenz 
markiert werden können. Der Algorithmus \ref{alg:analyze} beschreibt die Implementierung 
dieser Vererbung des Labels. Als Queue findet wieder die 
\texttt{de.unifrankfurt.faststring.analysis.util.UniqueQueue} Verwendung.

\begin{algorithm}[H]
	\caption{Vererbung des Labels}\label{alg:analyze}
	\begin{algorithmic}[1]
		\STATE $q \gets$ \texttt{new Queue($initialReferences$)}
		\STATE $g \gets$ der übergebene Datenflussgraph
		\WHILE {\texttt{not $q$.isEmpty()}}
			\STATE $r \gets q$\texttt{.remove()}
			\IF {\texttt{$r$.label} is not \texttt{null}}
				\STATE $def \gets$ \texttt{$r$.getDef()}
				\IF {\texttt{$def$.canProduce($r$.label)}}
					\STATE \texttt{labelConnectedRefs($def$)}
				\ENDIF
				\FOR {$use$ \bf{in} \texttt{$r$.getUses()}}
					\IF {\texttt{$use$.canUseAt($r$.label, $useIndex$)}}
						\STATE \texttt{labelConnectedRefs($use$)}
					\ENDIF
				\ENDFOR
			\ENDIF
		\ENDWHILE
		\RETURN $g$
	\end{algorithmic}
\end{algorithm}

Eine wichtige Rolle während des Vererbungsprozesses spielen dabei die Methoden 
\texttt{boolean Labelable.canProduce(TypeLabel)} und \\
\texttt{boolean Labelable.canUseAt(TypeLabel, int)}. Diese beiden Methoden bestimmen 
für eine Referenz ob ein Label auf eine Definition oder eine Benutzung vererbt werden kann.

\texttt{canProduce} sagt aus ob eine Instruktion eine optimierte Referenz definieren kann. 
Die Implementierungen der \texttt{canProduce} Methode sehen für die einzelnen \texttt{Labelable}
Subtypen wie folgt aus:

\begin{description}
	\item[NewNode] gibt immer \texttt{true} zurück
	\item[ConditionalBranchNode] gibt immer \texttt{false} zurück
	\item[MethodCallNode] Delegation an die Methode \texttt{canBeDefinedAsResultOf} des 
	übergebenen Label Objekts.
\end{description}

Die Methode \texttt{canUseAt} betrifft dagegen nur Benutzungen von Referenzen. Die Semantik
hinter der Methode ist ob diese Instruktion eine optimierte Referenz verwenden kann. Der Index 
stellt dabei den Parameter Index das an dem diese Referenz in dieser Instruktion verwendet wird. 
Die Implementierung dieser Methode sieht für die einzelnen \texttt{Labelable}
Subtypen wie folgt aus:

\begin{description}
	\item[NewNode] gibt immer \texttt{false} zurück
	\item[ConditionalBranchNode] gibt immer \texttt{true} zurück
	\item[MethodCallNode] Wenn die Methode nicht statisch und der Index 0 ist, wird an die Methode
	\texttt{canBeUsedAsReceiverFor} delegiert, andernfalls an die Methode \texttt{canBeUsedAsParamFor}.
\end{description}

Ist für eine Instruktion die Prüfung für die entsprechende Methode positiv, wird diese 
Instruktion an die Hilfsfunktion \texttt{labelConnectedRefs} weitergeleitet. Diese Methode 
ist im folgenden dargestellt:

\begin{algorithm}[H]
	\caption{labelConnectedRefs}\label{alg:labelConnRefs}
	\begin{algorithmic}[1]
		\STATE \texttt{$node$.setLabel($label$)}
		\FOR{$ref$ \bf{in} $node$\texttt{.getLabelableRefs()}}
			\IF{\texttt{$ref$.getLabel()} is not \texttt{null}}
				\STATE \texttt{$ref$.setLabel($label$)}
				\STATE \texttt{$q$.add($ref$)}
			\ENDIF 
		\ENDFOR
	\end{algorithmic}
\end{algorithm}

Wobei sowohl die Queue $q$, als auch das betreffende TypeLabel $label$ aus dem umgebenen Kontext
im Algorithmus \ref{alg:analyze} auch in dieser Funktion zur Verfügung stehen. 

\subsection{Umgang mit mehreren Labels}

Es ist möglich für die Analyse mehrere Labels für die Analyse zu verwenden. Diese lassen
sich dem \texttt{MethodAnalyzer} als \texttt{Collection} im Konstruktor zusammen 
mit der zu analysierenden Methode übergeben. Vor der Analyse werden zuerst alle Referenzen 
identifiziert von denen eine zu optimierende Methode verwendet wird. Diese Referenzen werden 
zu einer Menge an Referenzen vereint. Die auf diese Weise erzeugten Referenzen besitzen 
wie in \ref{ssec:umAlg} beschrieben jeweils das Label für das die Referenz erzeugt wurde.

Ein Problem ergibt sich, stellt man der Analyse mehrere Labels zur Verfügung, die allerdings
Optimierungen für dieselbe Methode des selben Typs beschreiben. Die erzeugten initial erzeugten 
Referenz-Objekte, welche die zu optimierende Methode verwenden, werden bei ihrer Erzeugung
mit dem entsprechenden Label markiert. Dieses Implementierungsdetail führt dazu, dass 
die entsprechenden Referenzen mit demjenigen Label markiert werden, welches zuerst verarbeitet wird.
Dieses wird immer das Label sein, welches einen geringeren Index in der Liste von Labels 
besitzt. 

Stellt man der Methode nun zwei unterschiedliche Labels zur Verfügung, die allerdings den
selben originalen Typen verwenden, ergibt sich ein anderes Problem, wenn innerhalb der analysierten 
Methode auf einer Referenz sowohl die eine als auch die andere Methode aufgerufen wird. Da das 
System jeder Referenz in einer Methode nur ein Label zuweisen kann, kann diese Referenz, welche 
beide Methoden der beiden TypeLabels verwendet, nur mit einem der TypeLabels markiert werden. 
In diesem Szenario wird wieder dasjenige Label für die Markierung verwendet, welches den niedrigeren
Index innerhalb der Liste besitzt.

Nachdem der Datenflussgraph initial erzeugt wurde, wird der Algorithmus wie in \ref{ssec:umAlg} 
beschrieben ausgeführt. Da Labels nur auf Knoten vererbt werden, gilt dass der entsprechende
Knoten dasjenige Label erhält, welches zuerst auf diesen Knoten vererbt wird.  

\subsection{Phi-Knoten}

$\phi$-Instruktionen innerhalb des Datenflussgraphen erfordern, aufgrund ihrer Eigenheiten,
eine besondere Behandlung. Da sie keine Instruktionen im eigentlichen Sinne darstellen, sondern
nur ein Zeiger alias für andere Referenzen darstellen, lassen sich diese Knoten auch nicht
optimieren. Obwohl sie selber nicht optimiert werden können, so lässt sich doch über diese 
Knoten hinweg ein Label vererben. Allerdings sollte unterschieden werden zwischen denjenigen
verbundenen Referenzen, für die eine Markierung tatsächlich sinnvoll und für welche eher
weniger sinnvoll ist. 

Daher sollten diejenigen Referenzen markiert werden, für die auch Optimierungspotenzial 
besteht. Wohingegen diejenigen Referenzen, für die keinerlei Optimierung notwendig 
ist, auch keine Markierung erhalten sollten. 

Wenn das System beim Vererben der Markierung einen $\phi$-Knoten als Instruktion verarbeitet,
wird dieser auf den $\phi$ Stapel gelegt und alle verbundenen Referenzen für die
weitere Verarbeitung vorgemerkt. Nachdem alle Referenzen verarbeitet wurden, wird dieser
Stapel durchlaufen. Für jede \texttt{PhiInstructionNode} in diesem Stapel wird entschieden
ob und mit welchem Label diese Instruktion markiert wird. 
\chapter{Transformation}
\label{ch:trans}

In diesem Kapitel sollen die Überlegungen und der Prozess der Bytecode Transformation,
auf Basis der Resultate aus dem Analyse Prozess, vorgestellt werden. Es wird zunächst 
auf die Beschaffung der nötigen Informationen eingegangen, um im Anschluss die Regeln,
nach denen Bytecode generiert oder manipuliert wird, erläutert.

Ziel der Transformation ist es zum Einen an den Grenzen der Bubble Konvertierungen zwischen den
Originalen und den optimierten Typen in den Sourcecode einzufügen. Zum Anderen müssen Uses
markierte Uses in entsprechende optimierte Versionen umgewandelt werden.  

\section{Lokale Variablen}
\label{sec:locals}

\subsection{Optimierte Variablen}

Um originale lokale Variablen im Bytecode nicht mit den optimierten Versionen zu 
überschreiben wurde eine Abbildung geschaffen, die jeder lokalen Variable ein Tupel 
zuweist: $l \rightarrow (L,l')$, wobei $l$ die originale lokale, $L$ ein Label und 
$l'$ die optimierte Variable für das Label $L$ darstellt. So ist sichergestellt, dass
optimierte und die entsprechende originale Referenz in zwei verschiedenen Lokalen 
geführt werden. Darüber hinaus ist diese Trennung wichtig, da die JVM die Plätze für
lokale Variablen typisiert und daher nicht an verschiedenen Stellen im Programm 
verschiedene Typen in derselben lokalen Variable gespeichert werden können.

\subsection{Variablen zu Value Numbers}

Da der IR mit den beinhalteten value numbers eine Abstraktion des eigentlichen Bytecodes
darstellt, fehlt auch jeglicher Bezug zu den eigentlichen lokalen Variablen, die von
einer spezifischen value number dargestellt wird. Darüber hinaus muss für Definition
einer Instruktion das \texttt{STORE} (schreibt die auf dem Stack liegende Referenz in 
die gegebene lokale Variable) und für alle Uses entsprechende \texttt{LOAD}s (ließt die 
gegebene lokale Variable) im Bytecode lokalisiert werden. Diese Informationen sind nötig,
da im Falle von Optimierungen, die für die entsprechende Instruktion erzeugt werden, 
die optimierten statt die originalen Referenzen geladen werden müssen.

Diese Informationen werden in der \texttt{InstructionNode} gehalten. Beim erzeugen 
eines solchen Objekts wird in der \texttt{InstructionNodeFactory} zum einen versucht
die lokalen zu den verwendeten value numbers zu erschließen und zum anderen die auf 
Position im Bytecode zu schließen an der die entsprechenden Werte auf den Stack gelegt 
werden.

Der IR, der aus einer class-Datei erzeugt wird besitzt ein privates Feld \texttt{localMap}
vom Typ \texttt{com.ibm.wala.ssa.IR.SSA2LocalMap}, welche in ihrer einzigen Implementierung
(der \texttt{com.ibm.wala.ssa.SSABuilder.SSA2LocalMap}) eine private Methode mit 
Signatur \texttt{int[] findLocalsForValueNumber(int, int)}, welche für eine gegebenen 
Bytecode Index und value number alle möglichen lokalen Variablen für diese value number 
an der gegebenen Stelle zurückgibt. Um diese Methode trotz aller Zugriffsbeschränkungen 
aufzurufen wurde eine Methode in \textit{Groovy} geschrieben um auf diese Methode 
zuzugreifen. Beim Suchen nach lokalen Variablen muss zwischen value numbers als Definition 
und als Use unterschieden werden. Das folgende Beispiel beschreibt das Problem bei 
Definitionen:

\begin{figure}[H]
	\begin{lstlisting}[language=Bytecode]
		INVOKEVIRTUAL org/example/SomeType.f()I // index 1
		ISTORE 5								// index 2
	\end{lstlisting}
	\caption{Lokale Variable für Definition}
\end{figure}

Die lokale Variable der Definition der \texttt{INVOKEVIRTUAL} Instruktion ist zum
Zeitpunkt des Methodenaufrufs noch nicht bekannt. Erst im Index 2 wird dieser Wert der 
lokalen Variablen 5 zugewiesen.

Um nun die Stellen zu finden an denen Variablen auf den Stack gelegt oder vom Stack 
gepoppt werden wurde eine einfache Stacksimulation eingeführt, wie sie in Algorithmus
\ref{alg:stack} zu sehen ist.

\begin{algorithm}[H]
	\caption{Simulation des Stacks}\label{alg:stack}
	\begin{algorithmic}[1]
		\STATE $size \gets$ Höhe des Stacks zum Zeitpunkt der Instruktion 
		\STATE $index \gets$ Index der betroffenen Referenz innerhalb des Stacks
		\STATE $bcIndex \gets$ Bytecode Index der betroffenen Instruktion
		\WHILE {\texttt{$actBlock$.getPredNodes() $= 1$}}
			\STATE \texttt{$actBlock \gets callGraph$.getBlockFor($bcIndex$)}

			\WHILE {\texttt{$bcIndex > actBlock$.getFirstInstructionIndex()}}
				\STATE $bcIndex \gets bcIndex - 1$
				\STATE $instruction \gets instructions[bcIndex]$
				\IF{\texttt{$instruction$.getPushedCount() $ > 0$}}
					\STATE $size \gets size - 1$
					\IF {index == size}
						\RETURN $bcIndex$ 
					\ENDIF
				\ENDIF
				\STATE \texttt{$size \gets size + instruction$.getPoppedCount()}
			\ENDWHILE	
		\ENDWHILE
		\RETURN $-1$ // kein Index gefunden
	\end{algorithmic}
\end{algorithm}

Dieser Algorithmus funktioniert für Definitionen, also das Suchen von \texttt{STORE}
Instruktionen, ähnlich. Der Unterschied liegt dabei ausschließlich im Inkrementieren 
(statt Dekrementieren) der $bcIndex$ Variable und dem umgekehrten Verhalten beim 
\textit{push} bzw. \textit{pop} von Werten auf bzw. vom Stack.

Diese Informationen werden in dem entsprechenden \texttt{InstructionNode} Objekt
gespeichert. Zu diesem Zweck besitzt dieser Typ drei Abbildungen ($\mathbb{N} \rightarrow 
\mathbb{N}$), die zum Zeitpunkt der Erstellung befüllt werden:

\begin{description}
	\item [\texttt{localMap}] bildet eine value number auf eine lokale Variable ab
	\item [\texttt{loadMap}] bildet eine lokale Variable auf einen Bytecode Index ab, an 
	dem diese Variable auf den Stack geladen wurde
	\item [\texttt{storeMap}] bildet eine lokale Variable auf einen Bytecode Index ab, an
	dem \texttt{STORE} diese Referenz in die entsprechende Variable schreibt
\end{description} 

\section{Bytecode Generierung}

Die Generierung von neuem und die Manipulation von bestehendem Bytecode wird von einer Instanz der
Klasse \texttt{MethodTransformer} vorgenommen. Als Eingabe dient eine Instanz vom Typ 
\texttt{TransformationInfo}, welche auf einem \texttt{AnalysisResult} basiert und 
Informationen über die verwendeten lokalen Variablen zur Verfügung stellt. Diese Informationen
sind zum einen die verwendeten Variablen, als auch die Abbildung von originalen Lokalen
zu optimierten.

Anhand dieser Informationen werden sowohl Konvertierungen als auch Optimierungen zu dem bestehenden
Bytecode hinzugefügt. Eine Konvertierung beschreibt dabei eine Transformation von einem originalen 
zu einem optimierten Typ bzw. von einem optimierten Typ zu einem originalen Typ. Eine Optimierung 
hingegen ersetzt eine Instruktion mit einer optimierten Variante. Diese Mechanismen sollen in dem 
folgenden Abschnitt vorgestellt werden.

\subsection{Informationen des TypeLabels}
\label{ssec:infoLabel}

Um Optimierungen und Konvertierungen auf bestimmten Typen umzusetzen werden verschiedene 
Informationen benötigt. Diese sind:

\begin{itemize}
	\item Wie der optimierte Type heißt
	\item Wie ein optimierter Typ aus einem Originalen erzeugt wird
	\item Wie die Signatur einer optimierten Variante einer Methode aussieht
\end{itemize} 


Diese Informationen werden über die \texttt{TypeLabel} Definitionen dem System zur Verfügung gestellt.
Zu diesem Zweck enthält das \texttt{TypeLabel} Interface, zusätzlich zu denen in \ref{sssec:typeLabel} 
vorgestellten, die folgenden Methoden: 

\begin{description}
	\item[\texttt{getOptimizedType()}] Gibt ein \texttt{Class} Objekt des optimierten Typs zurück
	\item[\texttt{getOriginalType()}] Gibt ein \texttt{Class} Objekt des originalen Typs zurück
	\item[\texttt{getCreationMethodName()}] Definiert den Namen der statischen Methode innerhalb des
	optimierten Typs, die ein neues Objekt des optimierten Typs erzeugt
	\item[\texttt{getToOriginalMethodName()}] Definiert den Namen der Methode, die auf einem Objekt
	des optimierten Typen aufgerufen werden kann um eine äquivalente Instanz des originalen Typen
	zu erzeugen
	\item[\texttt{getReturnType(MethodReference)}] gibt den Rückgabe Typ der gegebenen Methode im optimierten
	Typ zurück. Der Typ wird als \texttt{String} Repräsentation in der internen JVM Form zurück gegeben.
	(Beispiel: \texttt{Ljava/lang/String;})
	\item[\texttt{getParams(MethodReference}] gibt die Parameter der gegebenen Methode im optimierten
	Typ zurück. Die Parameter werden als \texttt{String} konkatenierte Liste umgeben mit Klammern 
	zurück gegeben. Die Typen der Parameter werden als interne JVM Form angegeben (Beispiel: (\texttt{II})).
\end{description}

\subsection{Konvertierung}

Konvertierungen dienen der Kompatibilität zwischen Code innerhalb und außerhalb der "Bubble". 
Dabei betreffen diese ausschließlich Referenzen im Datenflussgraphen. Formal lässt sich dieser
Vorgang wie folgt beschreiben. Die Funktion $label:E \rightarrow L$, wobei $L$ die Menge aller in dem
System vorhandenen Label ist und $E = R$ $\cup$ $I$, weißt jedem Knoten eine Label Markierung zu. Ein Knoten
ohne Markierung bekommt das Label \textit{KEIN\_LABEL} zugewiesen. Bei der Entscheidung ob 
zwischen einer Instruktion $i$ und einer Referenz $r$ eine Konvertierung nötig ist sind zwei 
Entscheidungen zu treffen:

\begin{enumerate}
	\item Handelt es sich bei der Instruktion um eine \textit{labelable} Instruktion 
	\item Ist es eine Def ($i \rightarrow r$) oder eine Use ($r \rightarrow i$) Kante innerhalb des
	Graphen
\end{enumerate} 

Ist die betrachtete Instruktion ohnehin nicht in der Lage mit einem Label markiert zu werden, so
muss nur die Referenz betrachtet werden und wenn $label(r) \neq$ \textit{KEIN\_LABEL}, dann muss eine 
Konvertierung an dieser Kante eingeführt werden. 

Handelt es sich bei der Instruktion allerdings um eine \textit{labelable} Instruktion, so trifft die Methode
\texttt{needsConversationTo(Label)} (s. Algorithmus \ref{alg:nct}) des Objekts auf den die Kante zeigt die 
Entscheidung ob eine Konvertierung nötig ist.    


\begin{algorithm}[H]
	\caption{needsConversationTo(Label)}\label{alg:nct}
	\begin{algorithmic}[1]
		\IF{\texttt{isSameLabel(label)}}
			\RETURN \texttt{false}
		\ENDIF
		\IF{\texttt{label} $\neq$ \texttt{null}}
			\RETURN $not$ \texttt{label.compatibleWith(this.label)}
		\ELSE
			\RETURN $not$ \texttt{this.label.compatibleWith(null)}
		\ENDIF 
	\end{algorithmic}
\end{algorithm}

Konvertierungen werden von Instanzen vom Typ \texttt{Converter} durchgeführt. Dabei wird zwischen Definitions-
und Use-Konvertierungen unterschieden für jeweils eigene Implementierungen der \texttt{Converter} 
Schnittstelle existieren. Der \texttt{Converter} nutzt dabei das \textit{Visitor} Muster, welches
für die einzelnen Instruktionsknoten implementiert wurde. Die jeweiligen \textit{visit}-Methoden erstellen  
auf der zur Verfügung gestellten \texttt{ConversionPatchFactory} die Shrike Patch Objekte um die Bytecode 
Manipulationen durchzuführen. 

Die \texttt{ConversionPatchFactory} dient der Erstellung von Shrike Patch Objekten und dem Hinzufügen dieser
in den Bytecode der aktuell verarbeiteten Methode. Zusätzlich muss bei der Instanziierung eines Objektes 
angegeben für welche Konvertierung diese Factory verwendet wird. Zu diesem Zweck besitzt der Typ zwei 
Attribute vom Typ \texttt{TypeLabel}:

\begin{description}
	\item[\texttt{from}] Der Typ den ein Objekt vor der Konvertierung besitzt
	\item[\texttt{to}] Der Typ in den ein Objekt konvertiert werden soll
\end{description}

Die Methoden, die zum Erzeugen eines Patches zur Verfügung stehen, sind die Folgenden:

\begin{description}
	\item{\texttt{createConversationAtStart(local)}} Erstellt eine Konvertierung am Anfang der Methode
	für die gegebene lokale Variable
	\item{\texttt{createConversationAfter(local, bcIndex)}} Erstellt eine Konvertierung nach dem gegebenen
	Bytecode Index für die gegebene lokale Variable. Die Variablen Angabe kann auch weggelassen werden, 
	dann wird der konvertierte Wert nicht in die entsprechende optimierte Variable gespeichert.
	\item{\texttt{createConversationBefore(local, bcIndex)}}  Erstellt eine Konvertierung vor dem gegebenen
	Bytecode Index für die gegebene lokale Variable. Die Variablen Angabe kann auch weggelassen werden, 
	dann wird der konvertierte Wert nicht in die entsprechende optimierte Variable gespeichert.
\end{description}

Soll eine Konvertierung für eine Definition ohne eine gegebene lokale Variable durchgeführt werden, so 
wird ausschließlich Methodenaufruf zur Konvertierung zum bzw. vom optimierten Typ aufgerufen. Wird eine 
lokale Variable angegeben so wird zunächst die Referenz auf dem Stack mittels der \texttt{DUP} Anweisung 
verdoppelt. Nach dem Aufruf der Konvertierung Methode wird die optimierte Referenz in die entsprechende 
optimierte Variable gespeichert. Konvertierungen die am Anfang der Methode durchgeführt werden (was 
ausschließlich für Konvertierungen die Übergabeparameter betreffen) verhalten sich genauso wie 
Konvertierungen denen eine Variable mitgegeben wurde, mit dem Unterschied, dass anstatt der \texttt{DUP} 
Anweisung ein \texttt{LOAD} für die entsprechende Variable verwendet wird um die Referenz für den 
Methodenaufruf zu verwenden.

Wird eine Konvertierung für einen Use benötigt, muss die originale Referenz bevor sie verwendet wird, wieder in
den originalen Type umgewandelt werden. Daher wird an der Stelle nachdem die betroffene Referenz auf den Stack
gelegt wurde der Methodenaufruf zum Umwandeln in den originalen Typ eingefügt. Ist die lokale Variable 
allerdings bekannt, wird die \texttt{LOAD} Instruktion, die die originale Variable lädt ersetzt durch ein
Laden der optimierten Variable um diese danach in einen originalen Typ zu konvertieren. 

\subsection{Optimierung}

Für jede \textit{labelable} Node wird, wenn sie mit einem Label markiert ist, eine Optimierung durchgeführt. 
Dafür sind 3 Schritte notwendig, diese werden im Folgenden beschrieben.

Primäres Ziel der Optimierungstransformation ist es, die originalen Methodenaufrufe durch die Optimierten zu 
ersetzen. Wie in \ref{subs:conventions} beschrieben, wird für eine Methoden Optimierung immer derselbe Name 
angenommen. In jedem Fall wird zunächst der Typ des Empfänger Objekts auf den des optimierten Typ gesetzt. 
Ist für die Parameter Liste im optimierten Typ keine Alternative angegeben, wird diejenige der originalen 
Methode verwendet. Der Rückgabe Typ wird ebenfalls durch das \textit{Typelabel} bestimmt. So wird das Ziel der 
\texttt{INVOKE} Instruktion auf den optimierten Typ gesetzt.

Um die Methode auf der optimierten Referenz aufrufen zu können, muss diese, statt der originalen Referenz, zum
Zeitpunkt des Methoden Aufrufs auf dem Stack vorhanden sein. Darüber hinaus müssen, wenn es sich bei den 
Parametern um gelabelte Referenzen handelt, auch die optimierten Parameter Referenzen, statt der originalen, 
auf dem Stack liegen. Daher müssen für optimierte Referenzen die \texttt{LOAD} Anweisungen, die die Referenzen
für den Methodenaufruf auf den Stack legen, die optimierte statt der originalen Variablen laden.

Ist der Rückgabewert eines optimierten Methodenaufrufs ebenfalls markiert und wird mittels einer 
\texttt{STORE} Anweisung in eine Variable gespeichert, so muss diese Anweisung die optimierte Referenz 
in die entsprechende optimierte Variable speichern. Daher wird die entsprechende Anweisung 
(wenn die Methode als \textit{producing} definiert ist) ersetzt durch eine \texttt{STORE} Anweisung, die 
die zurückgegebene Referenz in die entsprechende optimierte Variable speichert.

\section{Schwierigkeiten}

\subsection{Innere Archive (JARs)}

Um Klassen eines gegebenen Programms zu lesen und zu verarbeiten bietet WALAs Shrike die Klasse
\texttt{OfflineInstrumenter} an. Diese Klasse erwartet eine JAR Datei und ermöglicht daraufhin
eine Iteration über alle Klassen innerhalb dieses Archivs. 

Befinden sich nun Klassen dieser Anwendung verpackt in einem Archivs innerhalb dieses Archivs,
ist es dem \texttt{OfflineInstrumenter} nicht möglich diese Klassen zu finden und damit auch zu verarbeiten. 
Daher ist das System nicht in der Lage Klassen die nicht innerhalb des zu verarbeitenden Java-Archivs zu finden 
sind zu verarbeiten.

\subsection{Typisierung von generiertem Bytecode} 

Bei dem Erzeugen von Bytecode mit Shrike 

\chapter{Auswertung}

In diesem Kapitel soll das entwickelte System einer Auswertung unterzogen werden.
Im ersten Abschnitt soll zunächst das das verwendete Werkzeug \textit{JMH} sowie 
die allgemeine Durchführung der Tests beschrieben werden. Im zweiten Abschnitt 
folgen dann die einzelnen Tests sowie deren Auswertungen.

\section{Vorraussetzungen}

\subsection{Java Microbenchmarking Harness}

\subsection{Test Durchführungen}

\section{Benchmarks}

\subsection{Test 1}

\subsection{Test 2}

\subsection{Test 3}
\subsection{Test 4}

\chapter{Fazit}

\end{document}