\chapter{Werkzeuge}

In den folgenden Abschnitten sollen die verwendeten Werkzeuge kurz vorgestellt werden. Dabei handelt es 
sich zum einen um den Java Bytecode, als auch um die Software Bibliothek \textit{WALA}, auf deren API das 
von mir entwickelte System basiert. 

\section{Java Bytecode}

Die Plattformunabhängigkeit, die in Java geschriebenen Programmen zugesprochen wird, ist vorallem mit der 
Rolle der Java  Virtual Machine zu erklären. Java Programme werden in einen Zwischencode, den Java Bytecode, 
übersetzt, welcher von der System spezifischen JVM ausgeführt wird. Dabei ist Programmiersprache Java nicht 
die einzige in Bytecode übersetzbare Sprache. Es existieren neben den bekanntesten Scala, Jython, Groovy,
JavaScript noch viele weitere. Einmal in Bytecode übersetzt in diesen Sprachen geschriebene Programme auf 
jeder der Java Spezifikation entsprechenden JVM auführen. 
\\
Bytecode ist eine Sammlung von Instruktionen welche durch \textit{opcodes} von 2 Byte Länge definiert 
werden. Zusätzlich können noch 1 bis $n$ Parameter verwendet werden. Die Sprache ist Stack-orientiert, das 
bedeutet, dass von Operationen verwendete Parameter über einen internen Stack übergeben werden. Als Beispiel 
dient der folgende Bytecode:

\begin{lstlisting}[language=Bytecode]
	ICONST 5 	// legt den konstanten int Wert 5 auf den Stack 
	ILOAD 1		// läd die lokale integer Variable 1 und legt sie auf den Stack
	IADD 		// addiert die ersten beiden Werte auf dem Stack und legt das Ergebnis auf den Stack
	ISTORE 2	// speichert den Wert auf dem Stack in der Variable 2
\end{lstlisting} 

Dabei existiert der Stack nur als Abstraktion für den eigentlichen Prozessor im Zielsystem. Wie die 
jeweilige JVM den Stack in der Ziel Plattform umsetzt ist nicht definiert. Die Instruktionen lassen sich in 
folgende Kategorien einordnen:

\begin{itemize}
	\item Laden und Speichern von lokalen Variablen (\texttt{ILOAD}, \texttt{ISTORE})
	\item Arithmetische und logische ausdrücke (\texttt{IADD})
	\item Object Erzeugung bzw. Manipulation (\texttt{NEW}, \texttt{PUTFIELD})
	\item Stack Verwaltung (\texttt{POP}, \texttt{PUSH}) 
	\item Kontrollstruktur (\texttt{IFEQ}, \texttt{GOTO})
	\item Methoden Aufrufe (\texttt{INVOKEVIRTUAL}, \texttt{INVOKESTATIC})
\end{itemize}  


\section{WALA}

Bei \textit{WALA} handelt es sich um die "T.J. Watson Library for Analysis". Eine von ehemals von IBM 
entwickelte Bibliothek für die statische Codeanalyse von Java- und JavaScript Programmen. Das Framework 
übernimmt dabei das Einlesen von \textit{class} Dateien und stellt eine Repräsentation, die sogenannte 
\textit{Intermediate Representation}, des Bytecodes zur Verfügung. Diese IR stellt die zentrale 
Datenstruktur dar und soll in diesem Abschnitt detailiert beschrieben werden.
\\
Für die Manipulation des Bytecodes existiert innerhalb des Frameworks ein Unterprojekt, das diese Aufgabe 
übernimmt: Shrike. Im Zweiten Abschnitt soll diese API kurz vorgestellt werden.  

\subsection{IR}

Die \textit{Intermediate Representation} abstrahiert vom Stack basierten Bytecode zu einer \textit{single 
static assignment} Form

\subsection{Shrike}

Shrike ist ein Unterprojekt innerhalb des WALA Frameworks. Shrike übernimmt dabei das Lesen und das 
Schreiben von Bytecode aus bzw. in class Dateien. Dabei wird es zum einen beim Erstellen eines IR aus einer 
Methode verwendet, zum Anderen bietet es eine "Patch-based" API an um den Bytecode einer eingelesenen 
Methode zu verändern. Dies geschieht über das Einfügen von \texttt{Patch}es, welche über einen 
entsprechenden \texttt{MethodEditor} überall im Bytecode einer Methode eingefügt werden oder auch 
ursprüngliche Instruktionen komplett ersetzen. Zusätzlich enthält es einen \texttt{Verifier}, der erzeugten 
Bytecode überprüft, so dass ungültige Stack Zustände oder Typfehler noch während der Manipulation erkannt 
werden können. 
\\
In dem von mir entwickelten System werden alle Bytecode Manipulationen mit Hilfe von Shrike umgesetzt. 

