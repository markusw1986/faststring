\chapter{Einleitung}
\pagenumbering{arabic}
Die Verwendung von optimierten Datentypen beim Entwickeln eines Programms ist schwierig.
Die eingesetzten Implementierungen von Standarddatentypen decken zumeist allgemeine 
Fälle ab, sind aber nicht optimal für spezielle Szenarien. Zudem sind spezielle
optimierte Datentypen den Anwendern meist nicht oder zumindest nur begrenzt bekannt.
So wird bei der Entwicklung von Software der Datentyp verwendet oder zwischen den 
Datentypen gewählt, der dem Entwickler bekannt ist. Dabei ist nicht nur die fehlende 
Kenntnis über diese optimierten Typen ein Grund für das Übergehen ebendieser. Die 
Entscheidungsfindung bis hin zu Wahl einer Entscheidung für oder gegen einen speziellen
Typ verzögert die Arbeit während der Entwicklung von Systemen. 

Ein weiteres Problem ergibt sich bei der Betrachtung von Altsystemen. Bei bereits 
lang bestehenden Anwendung, besteht das Problem der Wartbarkeit von eingesetzten Datentypen. 
Durch die Weiterentwicklung von eingesetzten Bibliothek können Optimierungspotentiale
erst nach der aktiven Entwicklung von einzelnen Modulen entstehen. Hierbei ergibt sich
wieder das Problem, dass diese Potentiale bekannt und auch richtig umgesetzt werden.

Ziel dieser Arbeit ist es daher zu untersuchen, ob sich Programme durch Auswahl und 
Substitution von alternativen Datentypen automatisiert optimieren lassen. Zu diesem Zweck
soll ein System erstellt werden, dass mittels statischer Codeanalyse und automatischer
Transformation eine Optimierung eines Programms durchführt \footnote{Der Quellcode zu 
dem System (inklusive Tests) sowie die Benchmarks und Auswertungsskripte sind unter 
\texttt{http://github.com/wondee/faststring} zu finden.}.

Diese Untersuchung kann allerdings keine allumfassende Auswertung bereitstellen und wird 
sich auf die Optimierung der Laufzeit des \texttt{java.lang.String} Datentyps der 
\textit{Java} Plattform und der Programmiersprache \textit{Java} beschränken. Nach 
der Auswertung soll eine Aussage darüber möglich sein, ob diese Optimierungen möglich sind.
Daher sollen die beiden Hypothesen untersucht werden:
\\
\begin{enumerate}
	\item Der String Datentyp bietet viele Möglichkeiten der Optimierung
	\item Das automatische Ersetzen durch alternative Datentypen führt zu einem
	Performance Gewinn
\end{enumerate}

Im Rahmen dieser Arbeit soll ein System erstellt werden, dass String Operationen in einem
gegebenen Java Programm mit einer entsprechenden optimierten Version auf Basis des 
Java Bytecodes ersetzt. Dabei soll das System anhand statischer Code Analyse die Stellen
lokalisieren, an denen eine Optimierung angewendet werden kann, und mittels der Transformation des 
Bytecodes des Programms diese Optimierungen anwenden. Als Ergebnis wird ein lauffähiges
Programm erwartet, dass bei gleicher Eingabe eine geringere Ausführungszeit besitzt 
als das originale Programm.

Es existieren Arbeiten, die Optimierungen ebenfalls durch die Auswahl von alternativen
Datentypen anstreben \cite{coco, cham, brain}. Diese setzen entgegen der in dieser Arbeit 
verwendeten statischen auf dynamische Codeanalyse und beschränken sich auf die Auswahl von 
Container Typen, um diese zur Laufzeit zu ersetzen um eine Optimierung des Programms zu erreichen.

In Kapitel 2 werden die Werkzeuge und Grundlagen beschrieben, auf denen das in dieser Arbeit
entwickelte System aufbaut. Darauf folgt Kapitel 3, welches die optimierten String Typen vorstellt
die in dieser Arbeit entwickelt wurden. Kapitel 4 erläutert den Analyse Prozess des Systems, indem
die Datenstrukturen sowie der Algorithmus für die statische Code Analyse beschrieben wird.
In Kapitel 5 wird basierend auf den Analyse Ergebnissen die Transformation des Bytecodes 
beschrieben. Kapitel 6 widmet sich der Auswertung der Tests in Form von Benchmarks
und begründet diese Ergebnisse. Kapitel 7 zieht ein Fazit und beschreibt mögliche
Weiterentwicklungen des Systems.    