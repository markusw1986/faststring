\chapter{Analyse}

Das Folgende Kapitel beschreibt den Analyse Algorithmus, des von mir entworfenen Systems.
Im ersten Abschnitt sollen die verwendeten Datenstrukturen vorgestellt und beschrieben werden.
Der zweite Abschnitt beschreibt den eigentlichen Algorithmus.

\section{Datenstrukturen}

Für den Algorithmus wurden zwei Grundlegende Datenstrukturen entworfen. Der 
\textit{Datenflußgraph} repräsentiert den Datenfluß der Referenzen innerhalb 
einer Methode und wird im ersten Abschnitt vorgestellt. Zu optimierende Referenzen
werden in diesem Graphen mit sogenannten \textit{Labels} versehen. Dieser Datentyp
woll im zweiten Abschnitt beschrieben werden.

\subsection{Datenfluß Graph}

Eine auf einem IR basierende Methode wird vom System mittels eines Datenfluß
graphen repräsentiert. Dieser wird vor der eigentlichen Analyse aus einer gegeben
Methode und einer Menge an initialen Referenzen vom \texttt{DataFlowGraphBuilder}
erzeugt. Der Graph ist gerichtet und setzt sich aus zwei verschiedenen Knoten zusammen:

\begin{itemize}
	\item \texttt{Reference}: eine value number aus dem IR
	\item \texttt{InstructionNode}: eine Instruktion aus dem IR
\end{itemize}

Sei im Folgenden der Datenflußgraph $G = (V, E)$, $R$ die Menge aller \texttt{Reference} 
Knoten und $I$ die Menge aller \texttt{InstructionNode}s. 
\\
Im Graph gilt $\forall (x,y) \in V,  (x \in R \wedge y \in I) \vee (x \in I \wedge y \in X)$.
Eine Kante $i \in I, r \in R, (i, r)$ beschreibt eine \textit{Definition}, die aussagt, 
dass die Referenz $r$ durch die Instruktion $i$ definiert wird. Ein Kante 
$i \in I, r \in R, (r, i)$ ist eine \textit{Benutzung} (im folgenden \textit{Use}
genannt). 

\texttt{Reference} Knoten werden für jede betroffene value number erzeugt. Für
die Erstellung von \texttt{InstructionNode} Objekten steht die 
\texttt{InstructionNodeFactory} zur Verfügung, die für eine gegebene 
\texttt{SSAInstruction} eine entsprechende \texttt{InstructionNode} erstellt. Um
für diesselbe \texttt{SSAInstruction} immer dasselbe \texttt{InstructionNode} 
Objekt zu garantieren verwendet die Factory einen internen Cache, der eine 
Abbildung $SSAInstruction \rightarrow InstructionNode$ verwaltet und für jede 
\texttt{SSAInstruction} prüft ob für diese bereits eine \texttt{InstructionNode} 
erstellt wurde.

Die Erstellung eines \texttt{DataFlowGraph}s beginnt immer mit einer Menge an 
initialen \texttt{Reference} Objekten. Ausgehend von dieser Startmenge werden über 
das \texttt{DefUse}-Object des betroffenen IRs die Definition und alle Uses in den 
Datenflußgraphen eingefügt. Algorithmus \ref{alg:dfg} beschreibt die Erstellung des 
Graphen.

\begin{algorithm}[H]
	\caption{Erstellung des Datenflußgraphen}\label{alg:dfg}
	\begin{algorithmic}[1]
		\STATE $q \gets$ \texttt{new Queue($initialReferences$)}
		\STATE $g \gets$ \texttt{new DataFlowGraph()}
		\WHILE {\texttt{not $q$.isEmpty()}}
			\STATE $r \gets q$\texttt{.remove()}
			\IF {\texttt{not $g$.contains()}}

				\STATE $def \gets$ \texttt{$defUse$.getDef($r$)}
				\STATE $uses \gets$ \texttt{$defUse$.getUses($r$)}

				\STATE \texttt{$newInd$.add($def$)}
				\STATE \texttt{$newInd$.add($uses$)}

				\STATE \texttt{$r$.setDef(factory.create($def$))}

				\FOR{\texttt{$ins \in defUse$.getUses($r$)}}
					\STATE \texttt{$r$.addUse(factory.create($ins$))}
				\ENDFOR
				\FOR{$ins \in newIns$}
					\STATE \texttt{$q$.add($ins$.getConnectedRefs())}
				\ENDFOR
				\STATE \texttt{$g$.add($r$))}
			\ENDIF
		\ENDWHILE
		\RETURN $g$
	\end{algorithmic}
\end{algorithm}

Jede \texttt{InstructionNode} besitzt eine Definition, die Nummer der Referenz 
die diese Instruktion erzeugt und eine Liste von Uses, die Nummern der Referenzen 
die es benutzt. Darüber hinaus noch Informationen zu Bytecode Spezifika, die 
im Kapitel \textit{Transformation} betrachtet werden.

Für verschiedene \texttt{SSAInstruction} typen existieren entsprechende 
\texttt{InstructionNode} Subtypen. Allerdings gibt es auch Typen die nicht
einer \texttt{SSAInstruction} zugeordnet werden können. Im Folgenden sollen die
wichtigsten Knotentypen vorgestellt werden. Es existieren darüber hinaus weitere
für die das System zur Zeit keine Unterstützung bietet, da es ausschliesslich für 
String Typen und komplexe Objekte ausgelegt ist.

\subsubsection{MethodCallNode}

Eine \texttt{MethodCallNode} repräsentiert einen Methoden Aufruf. Es besitzt, wenn
vorhanden, eine Definition, welche den Rückgabewert repräsentiert, einen Receiver, 
wenn es keine statische Methode ist und eine Liste an Parametern. Zusätzlich die 
aufgerufene Methode. 

\subsubsection{ContantNode}

Dieser Knoten Typ stellt eine Konstanten Definition dar. Er besitzt ausschließlich 
die Defintion, welcher Referenz diese Konstante zugewiesen wird. Für diesen Typ
existiert keine entsprechende \texttt{SSAInstruction}.

\subsubsection{ParameterNode}

Die \texttt{ParameterNode} stellt eine Definition eines Parameters der Methode dar.
Wird eine Variable innerhalb der Methode als Parameter in der Methoden Signatur
deklariert, wird deren Definition als \texttt{ParameterNode} im Datenflußgraphen
repräsentiert. Für diesen Typ existiert keine entsprechende \texttt{SSAInstruction}.

\subsubsection{NewNode}

Dieser Typ entspricht einer \texttt{NEW} Anweisung, die ein neues Objekt eines 
gegebenen Typen erstellt. Es besitzt eine Definition und den Typ des instanziierten
Objekts. 

\subsubsection{ReturnNode}

\texttt{ReturnNode} Typen sind \texttt{RETURN} Anweisungen. Die besitzen 
ausschließlich eine Referenz als Use. Diejenige Referenz, die sie aus der Methode 
zurückgeben. Dieser Typ kann keine Definition darstellen.

\subsubsection{PhiNode}

Die \texttt{PhiNode} steht für eine $\phi$-Instruktion aus dem IR. Sie besitzt eine
Referenz als Definition und 2 bis $n$ Uses.

\subsection{Label}

Das \textit{Label} entspricht einer Markierung, mit der Knoten in einem 
Datenflußgraphen versehen werden können. Dabei steht ein Label (oder 
\texttt{TypeLabel}, wie der Datentyp im System heißt) für einen Optimierten Typ.
Die Semantik hinter einem markierten Knoten ist, dass diese Referenz bzw. Instruktion
durch den entsprechenden Optimierten Typ ersetzt werden kann.

Es kann nicht für alle \texttt{InstructionNode}s ein Label gesetzt werden. Genauer
gesagt lassen sich ausschließlich für die Knotentypen \texttt{MethodCallNode}, 
\texttt{NewNode} und \texttt{PhiNode} ein Label setzen, da sich nur diese 
Instruktionen in einen optimierten Typ umwandeln lassen.

Das \texttt{TypeLabel} beinhaltet alle Regeln, die für die Verwendung eines
Optimierten Typen existieren. Dazu gehören

\begin{itemize}
	\item der Originale, sowie der Optimierte Typ
	\item die Methoden für die Optimierungen im optimierten Typ angeboten werden
	\item alle Methoden die darüber hinaus vom Optimierten Typ unterstützt werden
	\item Methoden, die den optimierten Typ als Rückgabewert zurückgeben
	\item kompatible Label
\end{itemize}

Dabei ist diese Liste bereits eine Abstraktion zu den Methoden, die das Interface 
besitzt. Im System lassen sich Label Definition auf 2 Arten erstellen:

\begin{enumerate} 
	\item Durch das Implementieren des Interfaces \texttt{TypeLabel}
	\item Durch das Erstellen einer \texttt{.type} Datei
\end{enumerate}

Zwar unterstützt das Kommandozeilen Tool zur Zeit nur die zweite Variante,
programmatisch lässt sich allerdings auch die erste Alternative umsetzen. Im 
Folgenden sollen die beiden Möglichkeiten zur Definition eines \texttt{TypeLabel}s
betrachtet werden.

\subsubsection{Das TypeLabel Interface}

Das Interface beinhaltet alle Methoden, die der Analyse- und Transformationsprozess
benötigt. In diesem Kapitel sollen zunächst nur die Methode betrachtet werden, die
für den Analyse Algorithmus verwendet werden. 

\begin{description}
	\item[\texttt{canBeUsedAsReceiverFor(MethodReference)}] Legt fest, ob eine 
	markierte Referenz als Receiver für den übergebenen Methodenaufruf dienen kann.
	\item[\texttt{canBeUsedAsParamFor(MethodReference,int)}] Bestimmt, ob eine 
	markierte Referenz als Parameter in dem gegebenen Methodenaufruf an der 
	entsprechenden Stelle (der \texttt{int} Parameter) verwendet werden kann.
	\item[\texttt{canBeDefinedAsResultOf(MethodReference)}] Sagt aus, ob die 
	gegebene Methode einen optimierten Typ zurückgeben kann. Dies impliziert, dass
	der Methodenaufruf selber auch markiert ist.
	\item[\texttt{findTypeUses(AnalyzedMethod)}] Gibt eine Menge an \texttt{Reference}
	Objekten zurück, auf denen innerhalb der gegebenen Methode eine der von 
	der Optimierung betroffenen Methode aufgerufen wird. Für diesen Algorithmus 
	existiert bereits eine Implementierung in der Klasse \texttt{BaseTypeLabel}.
	\item[\texttt{compatibleWith(TypeLabel)}] Gibt an, ob das übergebene Label 
	kompatibel mit diesem Label ist.
\end{description}

Alle diese Methoden werden von den \texttt{InstructionNode} Implementierungen 
verwendet. Wie genau das passiert wird im Abschnitt \textit{Algorithmus} beschrieben.

\subsubsection{Das .type Dateiformat}

Da das implementieren des Interfaces eher komplex ist, wurde für die einfachere 
Definition eines Types ein Datei Format entwickelt, welches von der Komplexität des
Interfaces abstrahieren soll. In dieser werden nicht die Regeln selbst, sondern 
die Fakten beschrieben, aus denen die Regeln hergeleitet werden können, beschrieben.

Aus einer Datei im \texttt{type} Format wird mittels eines internen Parsers ein
\texttt{TypeLabelConfig} Objekt erzeugt. 

\section{Algorithmus}

\subsection{Ziel}